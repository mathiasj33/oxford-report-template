\RequirePackage[l2tabu,orthodox]{nag}
\pdfsuppresswarningpagegroup=1
%%%%%%%%%%%%%%%%%%%%%%%%%%%%%%%%%%%%%%%%%%%%%%%%%%%%%%%%%%%%%%%
%% OXFORD THESIS TEMPLATE

% Use this template to produce a standard thesis that meets the Oxford University requirements for DPhil submission
%
% Originally by Keith A. Gillow (gillow@maths.ox.ac.uk), 1997
% Modified by Sam Evans (sam@samuelevansresearch.org), 2007
% Modified by John McManigle (john@oxfordechoes.com), 2015
% Modified by Mathias Jackermeier, 2024
%
% This version Copyright (c) 2024 Mathias Jackermeier
%
% Broad permissions are granted to use, modify, and distribute this software
% as specified in the MIT License included in this distribution's LICENSE file.
%

% I've (John) tried to comment this file extensively, so read through it to see how to use the various options.  Remember
% that in LaTeX, any line starting with a % is NOT executed.  Several places below, you have a choice of which line to use
% out of multiple options (eg draft vs final, for PDF vs for binding, etc.)  When you pick one, add a % to the beginning of
% the lines you don't want.


%%%%% CHOOSE PAGE LAYOUT
% The most common choices should be below.  You can also do other things, like replacing "a4paper" with "letterpaper", etc.

% This one will format for two-sided binding (ie left and right pages have mirror margins; blank pages inserted where needed):
%\documentclass[a4paper,twoside]{ociamthesis}
% This one will format for one-sided binding (ie left margin > right margin; no extra blank pages):
%\documentclass[a4paper]{ociamthesis}
% This one will format for PDF output (ie equal margins, no extra blank pages):
\documentclass[a4paper,nobind]{ociamthesis}

%%%%% SELECT YOUR DRAFT OPTIONS
% Three options going on here; use in any combination.  But remember to turn the first two off before
% generating a PDF to send to the printer!

% This adds a "DRAFT" footer to every normal page.  (The first page of each chapter is not a "normal" page.)
% \fancyfoot[C]{\emph{DRAFT Printed on \today}}  

% This highlights (in blue) corrections marked with (for words) \mccorrect{blah} or (for whole
% paragraphs) \begin{mccorrection} . . . \end{mccorrection}.  This can be useful for sending a PDF of
% your corrected thesis to your examiners for review.  Turn it off, and the blue disappears.
\correctionstrue


%%%%% BIBLIOGRAPHY SETUP
% Note that your bibliography will require some tweaking depending on your department, preferred format, etc.
% The options included below are just very basic "sciencey" and "humanitiesey" options to get started.
% If you've not used LaTeX before, I recommend reading a little about biblatex/biber and getting started with it.
% If you're already a LaTeX pro and are used to natbib or something, modify as necessary.
% Either way, you'll have to choose and configure an appropriate bibliography format...

\usepackage[%
backend=biber,
url=true,
mincrossrefs=999,
style=authoryear, % alphabetic, numeric
sorting=nyt, % default == nty, https://tex.stackexchange.com/questions/51434/biblatex-citation-order
maxnames=2,
minnames=1,
maxbibnames=99,
giveninits,
uniquename=init,
labeldateparts=true,
uniquelist=minyear,
date=year,
dashed=false]{biblatex}
\usepackage{stmaryrd}
%TC:ignore
%%% Change parentheses to square brackets %%%

%\ExecuteBibliographyOptions{parentracker=false}

% \makeatletter

% \newrobustcmd*{\parentexttrack}[1]{%
%   \begingroup
%   \blx@blxinit
%   \blx@setsfcodes
%   \blx@bibopenparen#1\blx@bibcloseparen
%   \endgroup}

% \AtEveryCite{%
%   \let\parentext=\parentexttrack%
%   \let\bibopenparen=\bibopenbracket%
%   \let\bibcloseparen=\bibclosebracket}

% \makeatother



%%% Add links to name %%%

\DeclareCiteCommand{\cite}
{\usebibmacro{prenote}}
{\usebibmacro{citeindex}%
	\printtext[bibhyperref]{\usebibmacro{cite}}}
{\multicitedelim}
{\usebibmacro{postnote}}

\DeclareCiteCommand*{\cite}
{\usebibmacro{prenote}}
{\usebibmacro{citeindex}%
	\printtext[bibhyperref]{\usebibmacro{citeyear}}}
{\multicitedelim}
{\usebibmacro{postnote}}

\DeclareCiteCommand{\parencite}[\mkbibparens]
{\usebibmacro{prenote}}
{\usebibmacro{citeindex}%
	\printtext[bibhyperref]{\usebibmacro{cite}}}
{\multicitedelim}
{\usebibmacro{postnote}}

\DeclareCiteCommand*{\parencite}[\mkbibparens]
{\usebibmacro{prenote}}
{\usebibmacro{citeindex}%
	\printtext[bibhyperref]{\usebibmacro{citeyear}}}
{\multicitedelim}
{\usebibmacro{postnote}}

\DeclareCiteCommand{\footcite}[\mkbibfootnote]
{\usebibmacro{prenote}}
{\usebibmacro{citeindex}%
	\printtext[bibhyperref]{ \usebibmacro{cite}}}
{\multicitedelim}
{\usebibmacro{postnote}}

\DeclareCiteCommand{\footcitetext}[\mkbibfootnotetext]
{\usebibmacro{prenote}}
{\usebibmacro{citeindex}%
	\printtext[bibhyperref]{\usebibmacro{cite}}}
{\multicitedelim}
{\usebibmacro{postnote}}

\DeclareCiteCommand{\textcite}
{\boolfalse{cbx:parens}}
{\usebibmacro{citeindex}%
\printtext[bibhyperref]{\printnames{labelname}%
\printtext{ [\printfield{year}\printtext{]}}}}
{\ifbool{cbx:parens}
{\bibcloseparen\global\boolfalse{cbx:parens}}
{}%
\multicitedelim}
{\usebibmacro{textcite:postnote}}


%%% Add comma between author and year %%%

\renewcommand*{\nameyeardelim}{\addcomma\space}

%%% Make et al. cursive %%%

%\usepackage{xpatch}

%\xpatchbibmacro{name:andothers}{%
%	\bibstring{andothers}%
%}{%
%	\bibstring[\emph]{andothers}%
%}{}{}
\DeclareSourcemap{
    \maps{
        \map{
            \pertype{inproceedings}
            \pertype{article}
            \step[fieldset=isbn, null]
            \step[fieldset=issn, null]
            \step[fieldset=editor, null]
            \step[fieldset=publisher, null]
            \step[fieldset=location, null]
            \step[fieldset=urldate, null]
        }
    }
}

% \renewcommand*{\labelalphaothers}{\textsuperscript{+}}
\setlength\bibitemsep{0.5\baselineskip}
% following two lines allow URL line breaks in long URLs
\setcounter{biburllcpenalty}{7000}
\setcounter{biburlucpenalty}{8000}

% This makes the bibliography left-aligned (not 'justified') and slightly smaller font.
\renewcommand*{\bibfont}{\small}

% Change this to the name of your .bib file (usually exported from a citation manager like Zotero or EndNote).
\addbibresource{references.bib}

%TC:endignore

% Uncomment this if you want equation numbers per section (2.3.12), instead of per chapter (2.18):
%\numberwithin{equation}{subsection}



%%%%% THESIS / TITLE PAGE INFORMATION
% Everybody needs to complete the following:
\title{This is a long\\and impressive title}
\author{Firstname Lastname\\ \fontsize{12}{12}\selectfont Supervised by Prof.\ A B\normalfont}
\college{}

% Master's candidates who require the alternate title page (with candidate number and word count)
% must also un-comment and complete the following three lines:
%\masterssubmissiontrue
%\candidateno{933516}
%\wordcount{28,815}

% Uncomment the following line if your degree also includes exams (eg most masters):
\renewcommand{\submittedtext}{Transfer report submitted for the degree of}
% Your full degree name.  (But remember that DPhils aren't "in" anything.  They're just DPhils.)
\degree{Doctor in Philosophy}
% Term and year of submission, or date if your board requires (eg most masters)
\degreedate{Trinity 2024}


%%%%% YOUR OWN PERSONAL MACROS
% This is a good place to dump your own LaTeX macros as they come up.

\DeclareMathOperator*{\argmax}{arg\,max}
\DeclareMathOperator*{\argmin}{arg\,min}

\newcommand{\R}{\mathbb{R}}

\newcommand*{\dash}{\,---\,}

\newcommand*{\citep}{\autocite}

%%%%% THE ACTUAL DOCUMENT STARTS HERE
\begin{document}

%%%%% CHOOSE YOUR LINE SPACING HERE
% This is the official option.  Use it for your submission copy and library copy:
\setlength{\textbaselineskip}{22pt plus2pt}
% This is closer spacing (about 1.5-spaced) that you might prefer for your personal copies:
%\setlength{\textbaselineskip}{18pt plus2pt minus1pt}

% You can set the spacing here for the roman-numbered pages (acknowledgements, table of contents, etc.)
\setlength{\frontmatterbaselineskip}{17pt plus1pt minus1pt}

% Leave this line alone; it gets things started for the real document.
\setlength{\baselineskip}{\textbaselineskip}


%%%%% CHOOSE YOUR SECTION NUMBERING DEPTH HERE
% You have two choices.  First, how far down are sections numbered?  (Below that, they're named but
% don't get numbers.)  Second, what level of section appears in the table of contents?  These don't have
% to match: you can have numbered sections that don't show up in the ToC, or unnumbered sections that
% do.  Throughout, 0 = chapter; 1 = section; 2 = subsection; 3 = subsubsection, 4 = paragraph...

% The level that gets a number:
\setcounter{secnumdepth}{2}
% The level that shows up in the ToC:
\setcounter{tocdepth}{2}


%%%%% ABSTRACT SEPARATE
% This is used to create the separate, one-page abstract that you are required to hand into the Exam
% Schools.  You can comment it out to generate a PDF for printing or whatnot.
%\begin{abstractseparate}
%	\input{text/abstract} % Create an abstract.tex file in the 'text' folder for your abstract.
%\end{abstractseparate}


% JEM: Pages are roman numbered from here, though page numbers are invisible until ToC.  This is in
% keeping with most typesetting conventions.
\begin{romanpages}

% Title page is created here
\maketitle
\setcounter{page}{1}  % reset page counter

%%%%% DEDICATION -- If you'd like one, un-comment the following.
%\begin{dedication}
%This thesis is dedicated to\\
%someone\\
%for some special reason\\
%\end{dedication}

%%%%% ACKNOWLEDGEMENTS -- Nothing to do here except comment out if you don't want it.
%\begin{acknowledgements}
% 	\input{text/acknowledgements}
%\end{acknowledgements}

%%%%% ABSTRACT -- Nothing to do here except comment out if you don't want it.
%\pdfbookmark{Abstract}{abstract}
%\begin{abstract}
%	\input{text/abstract}
%\end{abstract}

%%%%% MINI TABLES
% This lays the groundwork for per-chapter, mini tables of contents.  Comment the following line
% (and remove \minitoc from the chapter files) if you don't want this.  Un-comment either of the
% next two lines if you want a per-chapter list of figures or tables.
%\dominitoc % include a mini table of contents
%\dominilof  % include a mini list of figures
%\dominilot  % include a mini list of tables

% This aligns the bottom of the text of each page.  It generally makes things look better.
\flushbottom

% This is where the whole-document ToC appears:
{
	\hypersetup{hidelinks}
	\pdfbookmark{\contentsname}{toc}
	\tableofcontents
}

%\listoffigures
%	\mtcaddchapter
% \mtcaddchapter is needed when adding a non-chapter (but chapter-like) entity to avoid confusing minitoc

% Uncomment to generate a list of tables:
%\listoftables
%	\mtcaddchapter

%%%%% LIST OF ABBREVIATIONS
% This example includes a list of abbreviations.  Look at text/abbreviations.tex to see how that file is
% formatted.  The template can handle any kind of list though, so this might be a good place for a
% glossary, etc.
% First parameter can be changed eg to "Glossary" or something.
% Second parameter is the max length of bold terms.
\begin{mclistof}{List of Abbreviations}{3.2cm}
    \item[CS] Computer science
\end{mclistof}


% The Roman pages, like the Roman Empire, must come to its inevitable close.
\end{romanpages}


%%%%% CHAPTERS
% Add or remove any chapters you'd like here, by file name (excluding '.tex'):
\flushbottom
\chapter{Introduction}
\label{ch:introduction}
The template is configured to use \texttt{biblatex}\footnote{\url{http://www.ctan.org/pkg/biblatex}} for the bibliography. Entries can be cited with \citep{achiam2017Constrained} or in text with \cite{achiam2017Constrained}. To customise the citation style, modify the \texttt{biblatex} options in \texttt{main.tex}.


\chapter{Background}
\label{ch:background}
This is the background.

\chapter{Literature Review}
\label{ch:literature_review}
This is the literature review.





%%%%% REFERENCES

% JEM: Quote for the top of references (just like a chapter quote if you're using them).  Comment to skip.
%\begin{savequote}[8cm]
%The first kind of intellectual and artistic personality belongs to the hedgehogs, the second to the foxes \dots
%  \qauthor{--- Sir Isaiah Berlin \cite{berlin_hedgehog_2013}}
%\end{savequote}

\setlength{\baselineskip}{0pt} % JEM: Single-space References

{\renewcommand*\MakeUppercase[1]{#1}%
\printbibliography[heading=bibintoc]}


%% APPENDICES %%
% Starts lettered appendices, adds a heading in table of contents, and adds a
%    page that just says "Appendices" to signal the end of your main text.
%\setlength{\baselineskip}{\textbaselineskip}
%\startappendices
% Add or remove any appendices you'd like here:
%\include{text/appendix_benchmark}
%\include{text/appendix_hyperparameters}

\end{document}
